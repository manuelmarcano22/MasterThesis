\chapter{Discussion and Conclusions}\label{chap:conclu}
\thispagestyle{fancy}

To finish we come back to the four open questions about CVs in globular clusters discussed in the introduction. 

\section{Primordial CVs}

We have obtained the spectra of two new CVs, one of the spectra being the first one from a CV far from the cluster center. Increasing the sample size of CV spectra, and looking at the CVs in different locations in the cluster, increases the chances of detecting the predicted two distinct populations of CVs. Unfortunately, the data still remains scarce as the total number remains in single digits (6 know CV spectra out of a total 15 candidates). In the studied sample of spectra we did not find prove of detection of the two different population. By studying the emission and absorption features in their spectra, no major difference were found between the CVs. The only exception might be U22 as it presents a clear H$\alpha$ emission, but no $H\beta$ emission. This might be due to contamination of a close star in the field of view as the sharp absorption around the H$\alpha$ suggest.  

Studying the companion stars of the CVs can also give us a clue about primordial CVs, as dynamically formed ones are expected to have a bias towards having a more massive companion. On none of the obtained spectra we observed signature of a M star (TiO lines e.g \citep{Marsh_secondary_1990}). This suggest that the companion stars might possibly be a K type star ($0.54 - 0.9 \text M _\odot$ \citep{gray2005observation}. Knowing that the turnoff mass~\footnote{the turn off mass is the maximum mass on the main sequence. This can serve as a rough estimate of the maximum mass of main sequence stars in a globular clusters.} is $0.77 M_\odot$ for NGC 6397 \citep{de_marco_spectroscopic_2005}. This gives us a range of $\sim 0.5 - 0.7 \text M _\odot$. 

\section{Periods and dwarf novae}

As mentioned before H$\alpha$ emission line was used to detect radial velocity variations in the CVs. Since the period of U23 is known, we used it as a check if the period can be found from the short exposures taken by MUSE. However, since the method depends on the strength of the emission line (and the integration time required to detect it significantly), the short exposures available were not enough to be able to unambiguously determine the orbital parameters of the system.


\section{Magnetism and dwarf novae}
 As mentioned before it have been proposed that the majority of CVs in globular clusters are magnetic \citep{grindlay_magnetic_1999}. For NGC 6397, remarkably, all 4 previously identified CVs show prominent Helium II lines. These line are generally associated with magnetic and nova-like CVs \citep{echevarria_statistical_1988}. We did not detect any Helium II line in any of the two newly studied spectra (U10 and U22). This might suggest that they are not magnetic in nature. However, this is not conclusive as the strongest He II line is outside of the MUSE spectral range ($4686 \mathring{A}$). This means that the possibility that the two newly identified CVs are magnetic have not been completely ruled out. 

Regarding the dwarf novae, U22 show a moderate increase in magnitude compared to previous observation. This is not irrefutable evidence that U22 was observed during an outburst and photometric follow up to study the variability is needed for confirmation. If U22 is indeed a dwarf novae it would be the third one identify in NGC 6397 \citep{shara_erupting_2005}.  

Altogether we have demonstrated how an IFS like MUSE can be used to study the population of compact objects in globular clusters. Although we were limited by the very short exposure times, we were able to obtained new spectra and infer some properties about the CVs. These observations took place during the commissioning period and so were constrained by the overall commission goals. MUSE is now in regular observation time and among its future and current targets there is a number of other galactic globular cluster. Also improvement in the instrument are expected soon, like for example the use of adaptive optic and the narrow field mode. This means  a vast number of new data set that can be exploited to improve our understanding on the formation of compact objects in globular clusters. 

\begin{comment}

When U10 was first identity the X ray data suggested some magetism but we dont see any evidence in the specta. Tere

The X-ray spectral results suggest nine CVs, all with mod-
erately hard TB spectra and internal self-absorption. The in-
trinsic
N
H
, particularly for U10 (CV6), suggests that these sys-
tems may be dominated by magnetic C



variablity 

Discussion:

    Two population:
        Dynamically vs. primordial?
    Magnetism ?
    X-Ray ?


\end{comment}