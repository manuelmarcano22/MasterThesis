\chapter{Discussion and Conclusions}\label{chap:conclu}
\thispagestyle{fancy}

To finish we come back to the four open questions about CVs in globular clusters discussed in the introduction. 


\section{Primordial CVs}

In this work we studied the population of CVs in NGC 6397. Previous photometric and X-ray studies of this cluster have identified a sample of 15 CV candidates. In this work we located the candidates on the MUSE observations to examine their spectral signature and extend the sample of spectroscopically confirmed CV candidates. From the sample we were able to unambiguously identify 5 CVs via their bright H$\alpha$ line in emission, sign of an accretion disk. We have obtained the spectra of two new CVs, one of the spectra being the first one from a CV outside the cluster center. Increasing the sample size of CV spectra, and looking at the CVs in different locations in the cluster, increases the chances of detecting the predicted populations of CVs. For the CV candidates farther from the cluster center (with one exception) no H$\alpha$ line was detected. This suggest that these might belong to a different kind of 'faint' CV population. Their radial distribution and their weak H$\alpha$ emission might suggest that these are primordial CVs. CVs from primordial origins are expected to be near the minimum period and have their optical spectra dominated by the white dwarf as the companion has lost most of its mass to the white dwarf. Hydrogen emission is still expected but probably weakened and inside broad absorption lines, the signature of a white dwarf. The lack of bright hydrogen lines from these object also leaves the door open to the possibility that the white dwarfs are accreting from a helium dwarf companion. These are of great interest as gravitational waves sources for future missions like eLISA. Such types of CVs have never been identify in globular clusters although they are predicted to exist. Deeper observation with MUSE are needed to get spectra with good signal-to-noise ratio and correctly classify them as such. 

Studying the companion stars of the CVs can also give us a clue about primordial CVs, as dynamically formed CVs are expected to have a bias towards having a more massive companion. None of the obtained spectra show the signature of an M star (TiO lines e.g \citep{Marsh_secondary_1990}). This suggest that the companion stars might possibly be a K type star ($0.54 - 0.9 \text M _\odot$ \citep{gray2005observation}. Knowing that the turnoff mass~\footnote{the turn off mass is the maximum mass on the main sequence. This can serve as a rough estimate of the maximum mass of main sequence stars in a globular clusters.} is $0.77 M_\odot$ for NGC 6397 \citep{de_marco_spectroscopic_2005}, this gives us a range of $\sim 0.5 - 0.8 \text M _\odot$, leaving the possibility for the companion to be either a M type star or even a low mass white dwarf or helium dwarf.

\section{Periods}

The H$\alpha$ emission line was used to detect radial velocity variations in the CVs. Since the period of U23 is known, we used it as a check if radial velocity could be detected in the short exposures taken by MUSE. However, since the method depends on the strength of the emission line (and the integration time required to detect it significantly), the short exposures available were not enough to be able to unambiguously detect radial velocity shifts. This suggests that longer integration time ($> 25$ seconds) is needed to be able to determine the period of the CVs in NGC 6397 with the MUSE instrument. Exposures of a few minutes ($\sim 6 \text m$) with the data on two different nights combined would allow us to detect the radial velocity shift. With longer exposures of the globular clusters MUSE would be able to trace simultaneously the radial velocity shift of multiple objects allowing the possibility of determining the orbital parameters of binaries in globular clusters. This is of great interest as the number of CVs with known periods in globular clusters is very low.  

\section{Magnetism and dwarf novae}

It has been proposed that the majority of CVs in globular clusters are magnetic \citep{grindlay_magnetic_1999}. For NGC 6397, remarkably, all 4 previously identified CVs show prominent Helium II lines. These line are generally associated with magnetic and nova-like CVs \citep{echevarria_statistical_1988}. We did not detect any Helium II line in any of the two newly studied spectra (U10 and U22). This might suggest that they are not magnetic in nature. However, this is not conclusive as the strongest He II line is outside of the MUSE spectral range ($4686 \mathring{A}$). This means that the possibility that the two newly identified CVs are magnetic have not been completely ruled out. 

 Regarding the dwarf novae, U22 showed a moderate increase in magnitude compared to previous observations. Based on data from dwarf novae in the field we can estimate how likely it is that we have detected a CV during an outburst. About half of all known fields CVs are dwarf novae ($\sim 40 \%$) and most discovered undergo 2-5 mag outburst \citep{2001PASP..113..764D,warner_cataclysmic_2003}. With data from the AAVSO catalog with a sample of 21 field DNs studied for an interval of three years \cite{Szkody_21DN_1984} concluded that the probability of a DN being in quiescence at a random epoch is $\sim 85 \%$. Due to the close distance to NGC 6397 we can make the assumption that we would have detected any outburst during the MUSE observation of the cluster. Based on this assumption we should have detected $\sim 15 \%$ of the DN in NGC 6397. Even if we limit our sample size of potential DN to the 15 candidates we expect to see at least 1 CV in outburst during the observation. There are many caveats with the assumptions made as the estimated rates of DN are mainly empirical and can suffer from selection bias due to the fact that observations of bright and long period outbursts are easier to detect. 

This is not irrefutable evidence that U22 was observed during an outburst and photometric follow up to study the variability is needed for confirmation. If U22 is indeed a dwarf novae it would be the third one identified in NGC 6397 \citep{shara_erupting_2005} and contributes to the still small sample of DN outburst spectra (37 out of 1600 CVs \citep{2001PASP..113..764D}). The fact that only one possible DN in outburst was identified supports the claim made by \cite{shara_CVsDN_1996} that DNs are very rare in globular clusters, maybe due to the fact that the majority are magnetic and thus are less likely to undergo DN outburst. \\ 

\begin{comment}
%If magnetic they still can eruptt. Model for TV col and GK Per by Angelini and Verbunt 1989. But lead to shorter outburst and longer interval. This suports the idea that most ara magnetic in nature (poalrs or Intermedia polats. 

When U10 was first identity the X ray data suggested some magetism but we dont see any evidence in the specta. Tere

The X-ray spectral results suggest nine CVs, all with mod-
erately hard TB spectra and internal self-absorption. The in-
trinsic
N
H
, particularly for U10 (CV6), suggests that these sys-
tems may be dominated by magnetic C



variablity 

Discussion:

    Two population:
        Dynamically vs. primordial?
    Magnetism ?
    X-Ray ?


\end{comment}
