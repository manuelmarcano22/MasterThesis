\chapter{Future Work}\label{chap:future}
\thispagestyle{fancy}

\section{Follow up Observation}

    Dr. Guillot MUSE proposal for the qLMXB
    
\section{Data analysis}
    
    \begin{itemize}
\item Optimal Spectra Extraction \citep{horne_emission_1986}
\item Study short term variability. Need to account for change in seeing.
\item Use the available 14 000 spectra and search for Compact objects. There might be some symbiotic or something. 
\end{itemize}

\section{Reproducibility}
\begin{itemize}
\item  Create a Docker container for the data reduction with necessary software and right version for the pipeline. Containers  define  a   computing   
        environment that    captures    dependencies.   In  container-based systems,    the results are the same    
        regardless  of  the host    system. Concinouns analysis when combine wih continous integration and source control repository

a   script  capable of  running the analyses    from    start   to  finish. The continuous  integration provider runs   
the latest  version of code in  the specified   Docker environment  without manual  intervention.       See \cite{Beaulieu-Jones056473} for a complete description  

\item Create an Amazon Machine Image to be able to run analysis with the Amazon Elastic Compute Cloud. Reducing of MUSE data needs to be done in a computer with large storage and RAM memory. Amazon's Elastic Compute Cloud provide on-demand access to large-scale computational resources. An example of something like this done in academia is the work of \cite{ragan-kelley_collaborative_2013}

Made use of a cloud computing environment, such as provided by the Amazon Web Service (AWS)

More recently in astrophysics the approach to do the anamylsi on the cloud have been proposed fr the CHILES projec for the Square kilometer arrat. 

See \cite{Dodson_SKAAmazon_2016} for the details. 


\end{itemize}

