\thispagestyle{empty}
\begin{Large}
\textbf{Abstract}
\end{Large}
\bigbreak
Globular clusters are very old groups of stars. Due to their age and the gravitational interactions dominating the dynamics of the clusters, they are home to a significant fraction of compact binaries. The formation
and evolution of these kinds of binaries is still not completely understood. Of special interest is the globular cluster NGC 6397 as it is the closest core collapsed cluster and has therefore been extensively studied
with instruments like Chandra, Hubble Space Telescope, and more recently in the optical with the Multi Unit Spectroscopic Explorer (MUSE), installed on the Very Large Telescope (VLT). Integral field spectrographs, like MUSE, have many advantages compared to traditional long slit spectroscopy, as spectra are obtained for every pixel and thus every object in the large field of view $(1'$ x $1')$. Here we present analysis of the compact binary population in NGC 6397 taken with MUSE. The goal is to further understand the characteristics of the proposed bimodal population of cataclysmic variables in the cluster, which have been suggested to be of primordial and dynamically formed origin. In this work we were able to spectrocopically confirmed two new CV candidates as well  as retrieve the spectra of three previously identified CVs. Spectral analysis on the extracted spectra allow us to estimate the mass ratio for a sample of the identified CVs. We also were able to compare the magnitude in the R band with previous observation of NGC 6397. From the spectral emission and absorption lines were were also able to estimate the type of the possible companion and infer the magnetic nature of the CVs. We specifically searched for Helium II lines as signature of magnetism, and Titanium Oxide lines from a M type companion. None of these features were found in the spectra. In conclusion, we proved the capability of MUSE as a good tool to study the population of compact objects in globular clusters. The large field of view of MUSE and short exposure times needed to obtained good quality spectra can help to study the short term variability and solve the problem of lack of data about CVs and other compact objects in globular clusters. 

\vspace*{\stretch{1}} % centrage vertical
\clearpage
