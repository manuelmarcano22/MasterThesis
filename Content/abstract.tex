\thispagestyle{empty}
\begin{Large}
\textbf{Abstract}
\end{Large}
\bigbreak
Globular clusters are very old groups of stars. Due to their age and the gravitational interactions dominating the dynamics of the clusters, they are home to a significant fraction of compact binaries. The formation
and evolution of these kinds of binaries is still not completely understood. Of special interest is the globular cluster NGC 6397 as it is the closest core collapsed cluster and has therefore been extensively studied
with instruments like XMM, Chandra, Hubble Space Telescope, and more recently in the optical with the Multi Unit Spectroscopic Explorer (MUSE), installed on the Very Large Telescope (VLT). Integral field spectrographs, like MUSE, have many advantages compared to traditional long slit spectroscopy, as spectra are obtained for every pixel and thus every object in the large field of view $(1'$ x $1')$. Here we present analysis of the compact binary population in NGC 6397 taken with MUSE. The goal is to further understand the characteristics of the proposed bimodal population of cataclysmic variables in the cluster, which have been suggested to be of primordial and dynamically formed origin. Spectral analysis will allow us to examine the origin of these two populations.

\vspace*{\stretch{1}} % centrage vertical
\clearpage
